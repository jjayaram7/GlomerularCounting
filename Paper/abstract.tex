Measuring the glomerular number in the entire, intact kidney using non-destructive techniques is of immense importance in studying several renal and systemic diseases. Commonly used approaches either require destruction of the entire kidney or perform extrapolation from measurements obtained from a few isolated sections. A recent magnetic resonance imaging (MRI) method, based on the injection of a contrast agent (cationic ferritin), has been used to effectively identify glomerular regions in the kidney. In this paper, we propose a robust, accurate, and low-complexity method for estimating the number of glomeruli from such kidney MRI images.  The proposed technique uses a few expert-marked training images to obtain discriminative graphs that can distinguish the glomerular regions from the rest. These graphs are obtained using non-local sparse coding-based approaches and discriminative projection directions are obtained using graph-embedding techniques. For novel test images, patches are extracted and embedded in a low-dimensional space using the discriminative projection directions. Finally, glomerular regions are identified and counted by clustering the embedded test data. Results with a validation dataset show that the glomerular counts obtained with the proposed algorithm closely match the expert-marked ground truth. 